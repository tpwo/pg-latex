Rok 1647 był to dziwny rok, w którym rozmaite znaki na niebie i ziemi zwiastowały jakoweś klęski i nadzwyczajne zdarzenia.

 Współcześni kronikarze wspominają, iż z wiosny szarańcza w niesłychanej ilości wyroiła się z Dzikich pól i zniszczyła zasiewy i trawy, co było przepowiednią napadów tatarskich. Latem zdarzyło się wielkie zaćmienie słońca, a wkrótce potem kometa pojawiła się na niebie. W Warszawie widywano też nad miastem mogiłę i krzyż ognisty w obłokach; odprawiano więc posty i dawano jałmużny, gdyż niektórzy twierdzili, że zaraza spadnie na kraj i wygubi rodzaj ludzki. Nareszcie zima nastała tak lekka, że najstarsi ludzie nie pamiętali podobnej. W południowych województwach lody nie popętały wcale wód, które, podsycane topniejącym każdego ranka śniegiem, wystąpiły z łożysk i pozalewały brzegi. Padały częste deszcze. Step rozmókł i zmienił się w wielką kałużę, słońce zaś w południe dogrzewało tak mocno, że — dziw nad dziwy! — w województwie bracławskiem i na Dzikich polach zielona ruń okryła stepy i rozłogi już w połowie grudnia. Roje po pasiekach poczęły się burzyć i huczeć, bydło ryczało po zagrodach. Gdy więc tak porządek przyrodzenia zdawał się być wcale odwróconym, wszyscy na Rusi, oczekując niezwykłych zdarzeń, zwracali niespokojny umysł i oczy szczególniej ku Dzikim polom, od których łatwiej niźli skądinąd mogło się ukazać niebezpieczeństwo.

 Tymczasem na polach nie działo się nic nadzwyczajnego i nie było innych walk i potyczek jak te, które się odprawiały tam zwykle, a o których wiedziały tylko orły, jastrzębie, kruki i zwierz polny.

 Bo takie to już były te pola. Ostatnie ślady osiadłego życia kończyły się, idąc ku południowi, niedaleko za Czehrynem ode Dniepru, a od Dniestru, niedaleko za Umaniem, a potem już hen ku limanom i morzu step i step w dwie rzeki, jakby w ramę ujęty. Na łuku dnieprowym, na Niżu, wrzało jeszcze kozacze życie za porohami, ale w samych polach nikt nie mieszkał i chyba po brzegach tkwiły gdzieniegdzie „polanki“, jakoby wyspy wśród morza. Ziemia była de nomine Rzeczypospolitej, ale pustynna, na której pastwisk Rzeczpospolita Tatarom pozwalała, wszakże, gdy kozacy często bronili, więc to pastwisko było i pobojowiskiem zarazem.

 Ile tam walk stoczono, ilu ludzi legło, nikt nie zliczył, nikt nie spamiętał. Orły, jastrzębie i kruki jedne widziały, a kto z daleka usłyszał szum skrzydeł i krakanie, kto ujrzał wiry ptasie, nad jednem kołujące miejscem, to wiedział, że tam trupy lub kości niepogrzebione leżą... Polowano w trawach na ludzi, jakby na wilki lub suhaki. Polował kto chciał. Człek prawem ścigan chronił się w dzikie stepy, orężny pasterz trzód strzegł, rycerz przygód tam szukał, łotrzyk łupu, kozak Tatara, Tatar kozaka. Bywało, że i całe watahy broniły trzód przed tłumami napastników. Step to był pusty i pełny zarazem, cichy i groźny, spokojny i pełen zasadzek, dziki od Dzikich pól, ale i od dzikich dusz.

 Czasem też napełniała go wielka wojna. Wówczas płynęły po nim jak fale czambuły tatarskie, pułki kozackie, to chorągwie polskie lub wołoskie; nocami rżenie koni wtórowało wyciom wilków, głos kotłów i trąb mosiężnych leciał aż do Owidowego jeziora i ku morzu, a na Czarnym szlaku, na Kuczmańskim — rzekłbyś powódź ludzka. Granic Rzeczypospolitej strzegły od Kamieńca aż do Dniepru stanice i „polanki“ — i gdy szlaki miały się zaroić, poznawano właśnie po niezliczonych stadach ptactwa, które płoszone przez czambuły, leciały na północ. Ale Tatar, byle wychylił się z Czarnego lasu lub Dniestr przebył od strony wołoskiej, to stepem równo z ptakami stawał w południowych województwach.

 Wszelako zimy owej ptactwo nie ciągnęło z wrzaskiem ku Rzeczypospolitej. Na stepie było ciszej, niż zwykle. W chwili, gdy rozpoczyna się powieść nasza, słońce zachodziło właśnie, a czerwonawe jego promienie rozświecały okolicę pustą zupełnie. Na północnym krańcu Dzikich pól, nad Omelniczkiem, aż do jego ujścia, najbystrzejszy wzrok nie mógłby odkryć jednej żywej duszy, ani nawet żadnego ruchu w ciemnych, zeschniętych i zwiędłych burzanach. Słońce połową tylko tarczy wyglądało jeszcze z za widnokręgu. Niebo było już ciemne, a potem i step zwolna mroczył się coraz bardziej. Na lewym brzegu, na niewielkiej wyniosłości, podobniejszej do mogiły niż do wzgórza, świeciły tylko resztki murowanej stanicy, którą niegdyś jeszcze Teodoryk Buczacki wystawił, a którą potem napady starły. Od ruiny owej padał długi cień. Opodal świeciły wody szeroko rozlanego Omelniczka, który w tem miejscu skręca się ku Dnieprowi. Ale blaski gasły coraz bardziej na niebie i na ziemi. Z nieba dochodziły tylko klangory żórawi, ciągnących ku morzu, zresztą ciszy nie przerywał żaden głos.

 Noc zapadła nad pustynią, a z nią nastała godzina duchów. Czuwający w stanicach rycerze opowiadali sobie w owych czasach, że nocami wstają na Dzikich polach cienie poległych, którzy zeszli tam nagłą śmiercią w grzechu, i odprawują swoje korowody, w czem im żaden krzyż ani kościół nie przeszkadza. To też, gdy sznury, wskazujące północ, poczynały się dopalać, odmawiano po stanicach modlitwy za umarłych. Mówiono także, że one cienie jeźdźców, snując się po pustyni, zastępują drogę podróżnym, jęcząc i prosząc o znak krzyża świętego. Między nimi trafiały się i upiory, które goniły za ludźmi, wyjąc. Wprawne ucho z daleka już rozeznawało wycie upiorów od wilczego. Widywano również całe wojska cieniów, które czasem zbliżały się tak do stanic, że straże grały larum. Zapowiadało to zwykle wielką wojnę. Spotkanie pojedynczych cieniów nie znaczyło również nic dobrego, ale nie zawsze należało sobie źle wróżyć, bo i człek żywy zjawiał się nieraz i niknął jak cień przed podróżnymi, dlatego często i snadnie za ducha mógł być poczytanym.

 Skoro więc noc zapadła nad Omelniczkiem, nie było w tem nic dziwnego, że zaraz koło opustoszałej stanicy pojawił się duch czy człowiek. Miesiąc wychynął właśnie z za Dniepru i obielił pustkę, głowy bodiaków i dal stepową. Wtem niżej na stepie ukazały się i inne jakieś nocne istoty. Przelatujące chmurki przesłaniały co chwila blask księżyca, więc owe postacie to wybłyskiwały z cienia, to znowu gasły. Chwilami nikły zupełnie i zdawały się topnieć w cieniu. Posuwając się ku wyniosłości, na której stał pierwszy jeździec, skradały się cicho, ostrożnie, zwolna, zatrzymując się co chwila.

 W ruchach ich było coś przerażającego, jak i w całym tym stepie, tak spokojnym napozór. Wiatr chwilami podmuchiwał ode Dniepru, sprawując żałosny szelest w zeschłych bodiakach, które pochylały się i trzęsły, jakby przerażone. Nakoniec postacie znikły, schroniły się w cień ruiny. W bladem świetle nocy widać było tylko jednego jeźdźca, stojącego na wyniosłości.

 Wreszcie szelest ów zwrócił jego uwagę. Zbliżywszy się do skraju wzgórza, począł wpatrywać się w step uważnie. W tej chwili wiatr przestał wiać, szelest ustał, zrobiła się cisza zupełna.

 Nagle dał się słyszeć przeraźliwy świst. Zmieszane głosy poczęły wrzeszczeć przeraźliwie: „Hałła! hałła! Jezu Chryste! ratuj! bij!“ Rozległ się huk samopałów, czerwone światła rozdarły ciemność. Tętent koni zmieszał się z szczękiem żelaza. Nowi jacyś jeźdźce wyrośli jakby z pod ziemi na stepie. Rzekłbyś burza zawrzała nagle w tej cichej złowrogiej pustyni. Potem jęki ludzkie zawtórowały wrzaskom strasznym, wreszcie ucichło wszystko, walka była skończona.

 Widocznie rozegrywała się jedna ze zwykłych scen na Dzikich polach.

 Jeźdźcy zgrupowali się na wyniosłości, niektórzy pozsiadali z koni, przypatrując się czemuś pilnie.

 Wtem w ciemnościach ozwał się silny i rozkazujący głos:

 — Hej tam! skrzesać ognia i zapalić!

 Po chwili posypały się naprzód iskry, a potem buchnął płomień suchych oczeretów i łuczywa, które podróżujący przez Dzikie pola wozili zawsze ze sobą.

 Wnet wbito w ziemię drąg od kaganka i jaskrawe padające z góry światło oświeciło wyraźnie kilkunastu ludzi, pochylonych nad jakąś postacią leżącą bez ruchu na ziemi.

 Byli to żołnierze ubrani w barwę czerwoną, dworską, i w wilcze kapuzy. Z tych, jeden siedzący na dzielnym koniu, zdawał się reszcie przywodzić. Zsiadłszy z konia, zbliżył się do owej leżącej postaci i spytał:

 — A co, wachmistrzu, żyje, czy nie żyje?

 — Żyje, panie namiestniku, ale charcze; arkan go zdławił.

 — Co zacz jest?

 — Nie Tatar, znaczny ktoś.

 — To i Bogu dziękować.

 Tu namiestnik popatrzył uważniej na leżącego męża.

 — Coś jakby hetman — rzekł.

 — I koń pod nim Tatar zacny, jak lepszego u chana nie znaleźć — odpowiedział wachmistrz. — A ot tam go trzymają.

 Porucznik spojrzał i twarz mu się rozjaśniła. Obok dwóch szeregowych trzymało rzeczywiście dzielnego rumaka, który, tuląc uszy i rozdymając chrapy, wyciągał głowę i poglądał przerażonemi oczyma na swego pana.

 — Ale koń, panie namiestniku, będzie nasz? – wtrącił tonem pytania wachmistrz.

 — A ty psiawiaro, chciałbyś chrześcijaninowi konia w stepie odjąć?

 — Bo zdobyczny...

 Dalszą rozmowę przerwało silniejsze chrapanie zduszonego męża.

 — Wlać mu gorzałki w gębę — rzekł pan namiestnik — pas odpiąć.

 — Czy zostaniemy tu na nocleg?

 — Tak jest, konie rozkulbaczyć, stos zapalić.

 Żołnierze skoczyli co żywo. Jedni poczęli cucić i rozcierać leżącego, drudzy ruszyli po oczerety, inni rozesłali na ziemi skóry wielbłądzie i niedźwiedzie na nocleg.

 Pan namiestnik, nie troszcząc się więcej o zduszonego męża, odpiął pas i rozciągnął się na burce przy ognisku. Był to młody bardzo człowiek, suchy, czarniawy, wielce przystojny, ze szczupłą twarzą i wydatnym orlim nosem. W oczach jego malowała się okrutna fantazya i zadzierżystość, ale w obliczu miał wyraz uczciwy. Wąs dość obfity i niegolona widocznie oddawna broda dodawały mu nad wiek powagi.

 Tymczasem dwaj pachołkowie zajęli się przyrządzaniem wieczerzy. Położono na ogniu gotowe ćwierci baranie; zdjęto też z koni kilka dropiów, upolowanych w czasie dnia, kilka pardew i jednego suhaka, którego pachoł wnet zaczął obłupywać ze skóry. Stos płonął, rzucając na step ogromne, czerwone koło światła. Zduszony człowiek począł zwolna przychodzić do siebie.

 Przez czas jakiś wodził nabiegłemi krwią oczyma po obcych, badając ich twarze, następnie usiłował powstać. Żołnierz, który poprzednio rozmawiał z namiestnikiem, dźwignął go w górę pod pachy; drugi włożył mu obuszek w dłoń, na którym nieznajomy wsparł się z całej siły. Twarz jego była jeszcze czerwona, żyły jej nabrzmiałe. Nakoniec przyduszonym głosem wykrztusił pierwszy wyraz:

 — Wody!

 Podano mu gorzałki, którą pił i pił, co mu widocznie dobrze zrobiło, bo odjąwszy wreszcie flaszę od ust, czystym już głosem spytał:

 — W czyich jestem ręku?

 Namiestnik powstał i zbliżył się ku niemu.

 — W ręku tych, co waści salwowali.

 — Przeto nie waszmościowie schwycili mnie na arkan?

 — Mosanie, nasza rzecz szabla, nie arkan. Krzywdzisz waść dobrych żołnierzów podejrzeniem. Złapali cię jakowiś łotrzykowie, udający Tatarów, których jeśliś ciekaw, oglądać możesz, bo oto leżą tam porżnięci jak barany.
