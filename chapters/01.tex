\chapter{Przykładowy rozdzial}
% You can label chapter to refer to it later in the text, e.g. in the
% later chapter.
\label{chpt:przykladowy-rozdzial}

Cała praca powinna składać się z rodziałów \emph{chapters}, które dzielą
się na podrozdziały \emph{sections}. Podrozdziały dodatkowo
posiadają swoje punkty \emph{subsections}. Cała ta struktura widoczna
jest w spisie treści. Oprócz tego LaTeX umożliwia na dzielenie punktów
podrozdziałów na poszczególne części \emph{subsubsections}, które przy
ustawieniach tego szablonu są niewidoczne w spisie treści. Wciąż mogą
być jednak przydatne do dzielenia dłuższych punktów podrozdziałów na
logiczne części.

\section{Przykładowy podrozdział}

\subsection{Przykładowy punkt podrozdziału}

\subsubsection{Rysunki}

\begin{figure}[!htb]
    \centering
    \includegraphics[width=0.25\linewidth]{przykladowy-rysunek}
    \caption{Logo książki o LaTeXu w Wikibooks
    \cite{book:latex}}
    \label{fig:przykladowy-rysunek}
\end{figure}

Rysunki i tabele należy zamieszczać zgodnie ze wzorem, zawsze podając:

\begin{itemize}
    \item wyśrodkowaną pozycję rysunku: \mono{{\textbackslash}centering};
    \item wymiary rysunku i nazwę pliku źródłowego;
    \item podpis pod rysunkiem wraz z cytowaniem, jeśli pochodzi
    z zewnętrznego źródła;
    \item etykietę, którą warto zacząć od np.\ \emph{fig}, jako że
    przestrzeń nazw zdefiniowanych w \mono{{\textbackslash}label} jest
    globalna.
\end{itemize}

Jako parametr w \mono{{\textbackslash}includegraphics} podajemy samą
nazwę pliku bez rozszerzenia. Plik z~rysunkiem należy umieścić
w folderze \emph{figures} lub innych skonfigurowanych w preambule.
Dobrym pomysłem jest używanie formatów wektorowych, żeby zachować wysoką
jakość rysunków przy wydruku. Niestety LaTeX ma bardzo ograniczone
wsparcie SVG, dlatego częstym wyborem jest konwertowanie grafik
wektorowych do formatu PDF.

Każdy rysunek powinien mieć odwołanie w tekście. Tu jako przykład
rysunku służy logo książki o LaTeXu \ref{fig:przykladowy-rysunek}, która
dostępna jest w Wikibooks.

\subsubsection{Tabele}

\begin{table}[!ht]
    \centering
    \small
    \caption{Tytuł tabeli}
    \label{tbl:etykieta-tabeli}
    \begin{tblr}{%
        hlines,%
        vlines,%
        row{1}={font=\bfseries},%
        column{1}={halign=c},%
    }%
        L.p. & Opis                                \\
        1    & Podstawowa tabela z dwoma kolumnami \\
    \end{tblr}
\end{table}


Paczka \emph{tabularray} pozwala na proste definiowanie podstawowych
i bardzo złożonych tabel. Każda tabela również powinna mieć odwołanie
w tekście. Tu jako przykład została zdefiniowana tabela
\ref{tbl:etykieta-tabeli}. Warto zauważyć, że tabele i rysunki
numerowane są niezależnie, więc mogą dzielić ten sam numer. Z tego
powodu, z kontekstu zawsze powinno wynikać, czy mowa o tabeli, czy
o~rysunku.

\subsubsection{Cytowania i wykaz literatury}

Pozycje bibliograficzne należy dodawać w pliku
\emph{config/bibliography.bib} używając standardu biblatex. Pojawią się
one w wykazie literatury dopiero po tym, jak zostaną zacytowane w tekście
przy użyciu \mono{{\textbackslash}cite}. Jako przykład możemy zacytować
wspomnianą już książkę o LaTeXu \cite{book:latex}.

\subsubsection{Równania}

Równania najprościej definiować jest przy pomocy wbudowanego w LateXa
środowiska \mono{{\textbackslash}equation}, które wspiera wstawianie
symboli w trybie matematycznym. Równania również można oznaczać
etykietami, tak żeby oznaczyć je w tekście, ale w ich przypadku nie jest
to wymagane. W tym przypadku możemy odnieść się do równania
\ref{eq:przykladowe-rownanie}.

\begin{equation} \label{eq:przykladowe-rownanie}
    \Delta p = K_s (\frac{1}{T_0} - \frac{1}{T_s}) h
\end{equation}

gdzie:

\begin{conditions}
    \Delta p &  wielkość różnicy ciśnień $[Pa]$ \\
    K_s      &  współczynnik przeliczeniowy równy $3460$ \\
    T_0      &  temperatura powietrza zewnętrznego $[K]$ \\
    T_s      &  temperatura powietrza wewnętrznego $[K]$ \\
    h        &  odległość od płaszczyzny obojętnej $[m]$ \\
\end{conditions}

Każde równanie powinno używać symboli, które zostały jednoznacznie
zdefiniowane. Dobrym narzędziem jest do tego środowisko
\mono{{\textbackslash}conditions}, którego definicja znajduje się w tym
szablonie.

\subsubsection{Dobre praktyki}

LaTeX domyślnie używa większego odstępu między dwoma zdaniami niż między
wyrazami w~środku zdania, co jest popularną
anglosaską praktyką typograficzną spotykaną również w Polsce.
Alternatywne jest podejście francuskie, które używa takich samych
odstępów między wyrazami, jak między zdaniami. W LaTeXu to drugie
zachowanie można włączyć przy pomocy opcji
\mono{{\textbackslash}frenchspacing}.

Jeśli tego nie zrobimy, to należy pamiętać, żeby wszystkie skróty
stojące w środku zdania i zapisywane z kropką traktować w sposób
specjalny, tak by nie aktywowały tego zachowania. Da się to zrobić
poprzez zwykłe dodanie backslasha po kropce, po której odstęp ma się nie
zwiększyć. Np.\ skrót od \emph{na przykład} zapiszemy w ten sposób:
\mono{np.\textbackslash}

Alternatywnym sposobem jest dodanie twardej spacji, co można zrobić
przez podmianę spacji na tyldę \textasciitilde, ale powinna być ona
używana tylko tam, gdzie faktycznie jest potrzebna, czyli w~praktyce do
eliminowania sierotek typograficznych na końcach wersów.

\subsubsection{Podsumowanie}

Ten rozdział tłumaczy podstawowe zasady pracy z szablonem na żywych
przykładach i dlatego warto zobaczyć go zarówno w formie źródłowej jak
i jako wygenerowany PDF. Oprócz tego warto przejrzeć pliki
konfiguracyjne, gdzie poszczególne ustawienia opatrzone są dodatkowymi
komentarzami.
